\begin{recomendations}
Se identifican varias l\'ineas de investigaci\'on y experimentos a realizar para verificar el alcance y eficacia de la propuesta y su implementaci\'on en AutoGOAL.
\begin{itemize}
    \item Investigar en profundidad  como la propuesta pudiera adaptarse a sistemas que no utilizan algoritmos evolutivos para definir su espacio de b\'usqueda tales como los sistemas bayesianos o de aprendizaje por refuerzo.
    \item Probar la implementaci\'on utilizando algoritmos de Aprendizaje No Supervisado.
    \item Replicar las pruebas realizadas con AutoGOAL en \cite{estevez2020solving} utilizando como m\'etrica adicional al tiempo de entrenamiento y e investigar los resultados obtenidos. 
    %Verificar si la soluc\'on original de AutoGOAL esta incluida o si se encontraron variantes m\'as veloces. Comprobar que el desempe\~no de las  nuevas variantes  perfomen satisfactoriamente sobre el conjunto de pruebas.
    \item Mejorar el algoritmo multiobjetivo en el que se basa la propuesta. NSGA-II sufre un defecto caracter\'istico de muchos algoritmos multiobjetivos cuya rendimiento se ve crecientemente afectado mientras se eleva el n\'umero de m\'etricas. Es posible mejorar el sistema utilizando como base algoritmos evolutivos multiobjetivos por descomposici\'on tales como NSGA-III (\cite{deb2013evolutionary}) que no sufre de este defecto.
    
    \item Crear nuevas m\'etricas que analizen \textit{interpretabilidad} y \textit{resistencia} a perturbaciones de los datos. Utilizara junto a m\'etricas de relevancia en corpus de datos relevantes.
\end{itemize}

\end{recomendations}
