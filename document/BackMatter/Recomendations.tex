\begin{recomendations}
Siguiendo la propuesta de AutoML se identifican varias l\'ineas de investigaci\'on y experimentos a realizar para verificar el alcance y eficacia de la propuesta:
\begin{itemize}
    \item Investigar en profundidad en que como la propuesta puede adaptarse a sistemas que no utilizan algoritmos evolutivos para definir su espacio de b\'usqueda tales como los sistemas bayesianos o de aprendizaje por refuerzo.
    \item Replicar las pruebas realizadas en el paper de AutoGOAL utilizando como m\'etrica adicional al tiempo de entrenamiento y comparar las soluciones obtenidas. Verificar si la soluc\'on original de AutoGOAL esta incluida o si se encontraron variantes m\'as veloces. Comprobar que el desempe\~no de las  nuevas variantes  perfomen satisfactoriamente sobre el conjunto de pruebas.
    \item Mejorar el algoritmo multiobjetivo de la propuesta. NSGA-II sufre un defecto caracter\'istico de muchos algoritmos multiobjetivos cuya complejidad computacional crece polinomialmente con el n\'umero de m\'etricas. Es posible mejorar el sistema utilizando algoritmos evolutivos multiobjetivos por descomposic\'on tales como NSGA-III que no sufre de este defecto.
\end{itemize}

\end{recomendations}
