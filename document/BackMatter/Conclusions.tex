\begin{conclusions}
    La adici\'on de optimizaci\'on multiobjetivo a problemas AutoML incrementa exponencialmente sus usos. Los usuarios obtienen una forma efectiva de dirigir la b\'usqueda sin tener que participar activamente en esta. Problemas previamente sin resoluci\'on por sistemas AutoML, pueden ser atacados.

    Separar las componentes del sistema tal que trabajen en espacios separados posibilita la adici\'o intuitiva y no invasiva de este m\'ecanismo a los sistemas de AutoML. Adem\'as la utilizaci\'on de algoritmos evolutivos multiobjetivo garantiza soluciones igualmente distribuidas sobre el frente de Pareto, no importa cuan extra\~na sea su forma.

    Autogoal Multiobjetivo, una adaptaci\'on de la propuesta, es muestra de integraci\'on sencilla. La implementaci\'on mantiene  la misma interfaz sencilla de AutoGOAL est\'andar que permite su f\'acil uso a usuarios de cualquier \'ambito de experiencia. Se invita a utilizar esta nueva caracter\'istca de AutoGOAL.

    Los experimentos muestran resultados prometedores y el beneficio que significa la optimizaci\'on en paralelo de m\'ultiples criterios incluso para problemas que solo utilizan una s\'ola m\'etrica.
    Abre interrogantes interesantes sobre como se afectar\'ian experimentos previamente realizados con AutoGOAL despu\'es de la adici\'on de una m\'etrica adicional como tiempo de entrenamiento o complejidad del flujo.


    % El incremento exponencial en control que ofrece multiobjetvo para describi los prolemas, es cuesti\'on de esperar para encontrarla en los dem\'as sistemas AutoML.


    % se crean herramientas que brindan mayor expresivadad a los usuarios permitiendo la soluci\'on y mejoramiento de problemas actuales del campo del Aprendizaje Automatizado como la mitigaci\'on de sesgos. Adem\'as se obtienen beneficios inheretens de la optimizaci\'on multiobjetivo como el retorno de m\'ultples modelos igual de buenos.
    %El aprovechamiento de espacios separados donde trabajan las componentes de los sistemas AutoML crea la posiblidad de inclusi\'on de algoritmos multiobjetivos sin ser invasivos sobre los framework ya existentes. Ademas la utilizaci\'on de algoritmos evolutivos multiobjetivo permite la b\'usqueda efectiva de soluciones sobre el frente de Pareto.

    %La implementaci\'on basada en AutoGOAL es un ejemplo de cuan f\'acil es adaptar las soluciones ya existentes. La experimentaci\'on utilzando con AutoGOAL muestra resultados prometedores y abren todo un campo de problemas anteriores que se pueden probar nuevamente con esta caracter\'istica. La sencilla intefaz de AutoGOAL se mantiene sin cambios invitando a usuarios de cualquier \'ambito a comprobar esta nueva caracter\'isticas en sus problemas.
%  Los beneficios que promete la optimziaci\'on multiobjetivo son muchos es cuesti\'on de esperar que sistemas AutoML sigan este curso e implementen optimizaci\'on multiobjetivo.
\end{conclusions}
