\begin{conclusions}
    Al a\~nadir la habiliad de optimizar en paralelo para muchos objetivos en sistemas AutoML, se crean herramientas que brindan mayor expresivadad a los usuarios permitiendo la soluci\'on y mejoramiento de problemas actuales del campo del Aprendizaje Automatizado como la mitigaci\'on de sesgos. Adem\'as se obtienen beneficios inheretens de la optimizaci\'on multiobjetivo como el retorno de m\'ultples modelos igual de buenos.

    El aprovechamiento de espacios separados donde trabajan las componentes de los sistemas AutoML crea la posiblidad de inclusi\'on de algoritmos multiobjetivos sin ser invasivos sobre los framework ya existentes. Ademas la utilizaci\'on de algoritmos evolutivos multiobjetivo permite la b\'usqueda efectiva de soluciones sobre el frente de Pareto.

    La implementaci\'on basada en AutoGOAL es un ejemplo de cuan f\'acil es adaptar las soluciones ya existentes. La experimentaci\'on utilzando con AutoGOAL muestra resultados prometedores y abren todo un campo de problemas anteriores que se pueden probar nuevamente con esta caracter\'istica. La sencilla intefaz de AutoGOAL se mantiene sin cambios invitando a usuarios de cualquier \'ambito a comprobar esta nueva caracter\'isticas en sus problemas.
Los beneficios que promete la optimziaci\'on multiobjetivo son muchos es cuesti\'on de esperar que sistemas AutoML sigan este curso e implementen optimizaci\'on multiobjetivo.
\end{conclusions}
