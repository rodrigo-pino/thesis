\chapter{Detalles de Implementación y Experimentos}\label{chapter:implementation}

% Aqui hablar de todo en general (para este viersen)
% Donde es que se engancha mi implementaci\'on 
% Pintar un diagrama de esto 
% Explicar todas las partes del diagrama

Para la implementacion se defini\'o una nueva clase basado en la implementaci\'on de PGE de AutoGOAL. 
Se realizaron modificaciones sobre esta para obtener generalizar los resultados y poder trabajar con varias m\'etricas. Se adaptaron las funciones anteriores que solo optimzian para un solo objetivo y asum\'ian esto.

La clase NSPGE implementa NSGAII aprovechandose de la definicion de PGE como algoritmo genetico que funciona muy bien. Lo \'unico que cambia es realmente a la hora de ordenar los resultados que se utiliza una primera ordenacion por ranking y una segunda ordencaci\'on crowding distance.

Crowding distance require que los datos est\'en normalizados para esto se implemento feature scaling y se lleva por defecto todos los puntos en el rango de 0 a 1.

Primeramente se hace un recorrido por las soluciones y se busca los vecotes que dominan a otros y se van ordenando en una lista, donde en la primera posicion se encuentran los vectores a los que no domina nadie, en la segunda posicion los vectores a los que al menos alguien domina.

Luego se devuelven todos los frentes en orden hasta llegar a los N flujos requeridos. Si es necesario picar uno de los frentes estos se ordenan usando crowding distance.

% Y despues de las pruebas

% Probar cuando se usan varias m\'etricas (para este s\'abado)

% Plotear como se acerca al frente de Pareto
