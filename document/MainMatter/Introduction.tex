\chapter*{Introducción}\label{chapter:introduction}
\addcontentsline{toc}{chapter}{Introducción}
En la sociedad actual el Aprendizaje Autom\'atico tiene una presencia cada vez m\'as creciente en la vida cotidiana. Ha encontrado caso de uso donde los algoritmos convencionales resultan insuficientes, provocando su adopci\'on masiva por diversos campos de la ciencia y la t\'ecnica.
Se pueden encontrar algoritmos de Aprendizaje Autom\'atico en la medicina para la detecci\'on preventiva del c\'ancer \brackcite{kourou2015machine}, en el dise\~no de carros aut\'onomos \brackcite{badue2021self}, e incluso en el arte, donde son capaces de generar im\'agenes utilizando una descripci\'on textual \brackcite{xu2018attngan}.
% Su flexibilidad y capacidad de resolver problemas permite que tenga una amplia gama de aplicaciones en diversos dominios. La detecci\'on temprana de enfermedades \brackcite{kourou2015machine}, recomendaci\'on de noticias \brackcite{liu2010personalized} y conducci\'on aut\'onoma de veh\'iculos \brackcite{badue2021self} son ejemplos de su variado rango de aplicaciones.

Las popularidad del Aprendizaje Autom\'atico ha provocado un incremento en la demanda al punto que el n\'umero de expertos en el tema ya no pueden suplir. Adem\'as el dise\~no de un modelo de Aprendizaje Autom\'atico es un proceso que toma tiempo pues seleccionar los modelos e hiperpar\'ametros adecuados require abundante prueba y error.
Estas dificultades impulsan el estudio del Aprendizaje de M\'aquina Automatizado (\textit{Automated Machine Learning}, AutoML), una rama dedicada a facilitar el proceso de dise\~no de modelos de Aprendizaje Autom\'atico.

Los programas que resuelven el problema de Aprendizaje Automatizado son conocidos como sistemas AutoML y var\'ian en la parte de trabajo del experto que desean automatizar: desde la selecci\'on autom\'atica de hiperpar\'ametros \brackcite{feurer2019hyperparameter} a la selecci\'on de modelos \brackcite{thornton2013auto} o ambos  simult\'aneamente \brackcite{thornton2013auto}. El objetivo general de estos sistemas es producir una selecci\'on tal que el modelo final maximice respecto a cierta m\'etrica o criterio de evaluaci\'on. 
% Usualmente esta m\'etrica representa la relevancia de los resultados obtenidos.

% %la recomendaci\'on de noticias basadas en nuestros gustos online \brackcite{liu2010personalized},
% en redes sociales como moderadores de contenido a trav\'es de la detecci\'on de sentimientos de odio en un texto \brackcite{fortuna2018survey} 
% e incluso en el arte, donde sistemas producen im\'agenes a partir de descripciones textuales \brackcite{xu2018attngan}.
%El Aprendizaje Autom\'atico (\textit{Machine Learning} en ingl\'es) se encuentra en el centro de todas estas aplicaciones.

% TODO: Esta oracion necesita cita?

En el contexto actual, existen problemas para los cuales ya no es suficiente que un modelo de Aprendizaje Autom\'atico obtenga resultados relevantes pues se requiere adem\'as un buen desempe\~no respecto a criterios adicionales. En el campo del Aprendizaje de M\'aquina Automatizado esto implica la necesidad de incrementar la expresividad para describir los problemas con tal de obtener mayor control sobre las soluciones que estos generan. Entre las propuestas para lograr esto se encuentra la adici\'on de optimizaci\'on simult\'anea para m\'ultiples criterios de evaluaci\'on.   

\section*{Motivaci\'on}
% Los modelos de Aprendizaje de M\'aquina forman cada vez m\'as y m\'as parte de la vida diaria, aplic\'andose en situaciones directamente  relacionadas con la vida de los seres humanos. 
% TODO: Reescribir este parrafo para que se entienda que se sigue hablando de Multiobjetivo, y de xq es imortante
Un problema conocido y estudiado es el sesgo en los flujos de Aprendizaje de M\'aquina. Existen casos donde ciertas caracter\'isticas de los datos pueden provocar que las predicciones de un modelo se encuentren parcializadas en favor de una mayor\'ia  \brackcite{mehrabi2021survey} afectando su capacidad de predicci\'on. Esto tiene mayor importancia cuando los datos representan seres humanos y la decisi\'on del modelo tiene un impacto directo o indirecto en la vida de un individuo o grupo. Se desea que el modelo no discrimine por el color de piel, g\'enero o regi\'on y a la vez sus resultados se mantengan relevantes. Es necesario entonces que el modelo sea capaz de poder optimizar tanto  relevancia como  nivel de justeza, un problema multiobjetivo.

% TODO: Poner referencia  a situaciones de interpretabilidad y de robustez
Hay situaciones donde se quieren evaluar los modelos en s\'i utilizando alguna m\'etrica adicional tales como interpretabilidad \brackcite{molnar2020interpretable} o robustez \brackcite{papernot2016distillation}. La interpretabilidad permite entender sin necesidad de ser un experto porque el flujo toma ciertas decisiones y la robustez indica cuan resistente es el modelo frente a perturbaciones en los datos de entrenamiento. 

La optimizaci\'on multiobjetivo en AutoML permite el uso de m\'etricas que no tienen sentido usarlas individualemente como \textit{precisi\'on} y \textit{recobrado}. 
 \textit{Precisi\'on} mide la proporci\'on entre los valores relevantes identificados y todos los valores identificados
 mientras que \textit{recobrado} mide la proporci\'on entre los valores relevantes identificados y todos los valores relevantes.
Estas m\'etricas cuando se usan por separado no representan la relevancia de un modelo pues basta para tener un recobrado perfecto seleccionar todos los elementos del conjunto de datos, y una precisi\'on casi perfecta escogiendo un pequeño conjunto de elementos.
\textit{F-score} relaciona ambas medidas y permite el uso de  par\'ametros para priorizar una o la otra seg\'un el investigador determine. 
Un sistema AutoML Multiobjetivo puede prescindir de m\'etricas como \textit{f-score} pues es capaz de optimizar en paralelo para todos los criterios que la componen.
%que pueda realizar una b\'usqueda optimizando en paralelo para ambos criterios y obtener soluciones que representen todas las maneras de parametrizar \textit{f-score} en una solo ciclo de ejecuci\'on.
%puede producir soluciones variadas cada una representativa de ejecutar el problema con el f-score variando los parametros.

\section*{Problem\'atica}
% No darle la vuelta al pollo
% Decir cual es el problema: Poder utilizar optimizacion multiobjetivo dentro de AutoGOAL
El problema multiobjetivo aplicado a sistemas de AutoML es una tem\'atica poco estudiada y con aplicaciones pr\'acticas en la actualidad.
Los sistemas AutoML de mayor reconocimento en el campo solo son capaces de optimizar para una sola m\'etrica. 

Los sistemas de AutoML utilizan espacios de b\'usqueda ca\'oticos y no diferenciables para los cuales se necesitan algoritmos agn\'osticos a esos espacios y resistentes a formas extra\~nas del frente de Pareto. No existe un algoritmo de optimizaci\'on multiobjetivo lo suficientemente general que se pueda aplicar a cualquier sistema de AutoML.
% La capacidad de optimizar paralelamente para varios objetivos no resulta trivial y es necesario la integraci\'on de un algoritmo que sea lo m\'as resistente a este espacio sin sacrificar la obtenci\'on de buenos resultados. En la literatura hay varios ejemplos de optimizaci\'on multiobjetivo pero ninguno aplicado directamente al campo del AutoML. Los algoritmos evoluacionarios han resultado ser los mejores para la optimizaci\'on multiobjetivo pues cumplen con estas caracter\'isticas.

%Los sistema de Aprendizaje de M\'aquina Automatizado basados en algoritmos evolutivos son los que pueden tener el espacio de decision m\'as amplio.
%Se propone añadir multi objetivo para optimizar respecto a varias m\'etricas de evaluaci\'on simult\'aneamente. Se quiere utilizar una implementaci\'on de algoritmos gen\'etico ya que han demostrado ser los mejores en la literatura.
%Al  funcionar los dos con algoritmos evoluacionarios se busca hacer una integraci\'on entre ambos.

% \section*{Hip\'otesis}
% La integraci\'on de sistema AutoML genetico con algoritmo genetico

\section*{Antecedentes}
% Darle vaselina a esto:
% - Partir de que exsite un marco de trabajo respecto a sistemas AutoML: AutoGOAL 
% - Partir de que hay un estudio sobre sesgos en inteligencia artificial
% - Por ende, se quiere trabajar en una propuesta que permita multiobjetivo

% TODO: Es necesario especificar a que colectivo me refiero
En el entorno del grupo de Inteligencia Artificial de la facultad de Matem\'aticas y Computaci\'on de la Universidad de La Habana se desarrolla diversas l\'ineas de investigaci\'on centradas en el Aprendizaje de M\'aquina Automatizado. AutoGOAL \brackcite{estevez2020solving} es una herramienta creada por el colectivo con el objetivo de resolver el problema AutoML Heterog\'eneo.

La entidad tambi\'en desarrolla otras l\'ineas donde se estudia la posible mitigaci\'on de sesgos en flujos producidos por sistemas AutoML \brackcite{consuegra2022intelligent}. 
%Esto requiere de poder optimizar para m\'etricas adicionales como justeza.

Este trabajo es una propuesta que pretende incrementar la expresividad con que se pueden definir problema se AutoML. que tienen los usuarios para la b\'usqueda de sus soluciones permitendo la utilizaci\'on de varias m\'etricas.
%AutoGOAL una herramienta de AutoML que utilza algoritmos gen\'eticos y permite genera 


 
\section*{Objetivo}
\subsection*{Objetivo General}
Dise\~nar una propuesta de optimizaci\'on multiobjetvo en el dominio de problemas de AutoML
\subsection*{Objetivos Espec\'ificos}
\begin{itemize}
    \item Estudiar el estado del arte sobre sistemas de Aprendizaje de M\'aquina Automatizado
    \item Estudiar conceptos y principios de la optimizaci\'on multiobjetivo y sus aplicaciones en el campo del Aprendizaje de M\'aquina Automatizado
    \item Identificar e implementar un algoritmo multiobjetivo que aproveche la estructura de AutoGOAL y su implementaci\'on de Evoluci\'on Gramatical Probabil\'istica.
    \item Experimentar y analizar los resultados de aplicar AutoGOAL Multiobjetivo a problemas conocidos de la literatura.
\end{itemize}

\section*{Estructura de la Tesis}
El resto del documento se encuentra organizado de la siguiente manera. En el cap\'itulo \ref{chapter:state-of-the-art} se estudian t\'ecnicas de AutoML y optimizaci\'on multiobjetivo que conforman el estado del arte, as\'i como ejemplos de sistemas AutoML que apliquen t\'ecnicas de optimizaci\'on multiobjetivo. En el cap\'itulo \ref{chapter:proposal} se desarrolla una propuesta para resolver el problema de AutoML Heterog\'eneo utilizando optimizaci\'on multiobjetivo. El cap\'itulo \ref{chapter:implementation} muestra una implementaci\'on de la propuesta utilizando AutoGOAL. En el cap\'itulo \ref{chapter:experiments} se muestran resultados obtenidos tras aplicar la propuesta en tres corpus de datos distintos utilizando optimizaci\'on multiobjetivo para dos pares de m\'etricas. 
Finalmente, en el cap\'itulo 5 se presentan las conclusiones de la investigaci\'on.
