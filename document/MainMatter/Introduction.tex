\chapter*{Introducción}\label{chapter:introduction}
\addcontentsline{toc}{chapter}{Introducción}

% Intro
% Hacer mejor camino de automl -> a una sola m\'etrica -> a la necesidad de varias metricas
% Empezar xq existen los sistemas automl
% - Poner en contexto
% - formar expertos cuesta mucho tiempo
% - Existe un proceso repetitivo
% - Reformular los ultimos dos p\'arrafos

El Aprendizaje Autom\'atico (conocido en la literatura como \textit{Machine Learning}, ML por sus siglas en ingl\'es) es una rama de la Inteligencia Artificial enfocada en entender y construir programas que ``aprendan'' y logren buenos resultados en un conjunto de tareas definidas (\cite{mitchell1990machine}). Un campo que ha tenido un desarrollo matem\'atico considerable desde su creaci\'on en la d\'ecada de los 50 pero no encuentra una aplicaci\'on pr\'actica generalizada hasta el nuevo milenio debido a lo costoso computacionalmente que era entrenar modelos de Aprendizaje Autom\'atico. En la d\'ecada del 90 se ve el nacimiento del Aprendizaje de M\'aquina Automatizado (\textit{Automated Machine Learning}, AutoML), un nueva rama de estudio enfocada en el proceso de automatizar parte del trabajo de los expertos en Aprendizaje Autom\'atico.

Un programa que resuelve el problema de Aprendizaje Automatizado se le conocen como sistemas AutoML y var\'ian en la parte de trabajo del experto que desean automotizar: desde la selecci\'on autom\'atica de hiperpar\'ametros (\cite{feurer2019hyperparameter}) a  la selecci\'on de modelos (\cite{thornton2013auto}) o ambos  simult\'anemante. El objetivo general de estos sistemas es producir una selecci\'on tal que el modelo final maximice lo m\'as posible respecto a cierta m\'etrica o criterio de evaluaci\'on. Usualmente esta m\'etrica representa la relevancia de los resultados obtenidos.

% Construir un flujo v\'alido de Aprendizaje de M\'aquina requiere, entre otras cosas, probar con diferentes algoritmos e hyperpar\'ametros. Un proceso automatizable utilizando sistemas de Aprendizajes de M\'aquinas Automatizado (\textit{Automated Machine Learning}, AutoML).
% Estos sistemas en general reciben como entrada un conjunto de datos y una m\'etrica de evaluaci\'on y retornan un modelo de Aprendizaje Autom\'atico que tenga el mayor rendimiento de acuerdo a dicha m\'etrica. Normalmente los criterios de evaluaci\'on utilizados miden la relevancia de las respuestas del modelo. 

En el contexto actual, existen problemas para los cuales ya no es suficiente que un modelo de Aprendizaje Autom\'atico obtenga resultados relevantes pues requieren que tenga tambi\'en un buen desempe\~no respecto a criterios adicionales. En el campo del Aprendizaje Automatizado esto significa que el usuario necesita mayor expresividad para describir su problema con tal de obtener mayor control sobre el espacio de b\'usqueda. Una de las propuestas para lograr esto es la optimizaci\'on sim\'ultanea para m\'ultipes criterios de evaluaci\'on  tal que los modelos de ML producidos tengan buen desempeño sobre estas.   

% Se buscan sistemas que adem\'as de relevantes sean interpretables, que usuarios inexpectos sean capaces de entender las decisiones que toma el programa; robustos, que ante perturbaci\'on en los datos no cambien considerablemente.
% Adem\'as salen a relucir problemas desconocidos como modelos de Aprendizaje de M\'aquina que producen resultados paracializados dependiendo a ciertas carecter\'isticas de los datos. Problema especialmete grave cuando los datos representan seres humanos, y las decisiones del sistema tiene un impacto en el individuo o grupo. No se desea que juzgue por el color de piel o g\'enero. Nace la necesidad de producir sistemas justos que consideren ciertos atributos como protegidos y minimicen el sesgo que esto puedan provocar en sus decisiones.
% No es posible obtener producir estos sistemas buscando obtener la mejor relevancia, es necesario utilizar otras m\'etricas y lograr que el sistema AutoML busque maximizar la mayor puntuaci\'on posible en estas. 

\section*{Motivaci\'on}
%  
Los modelos de Aprendizaje de M\'aquina forman cada vez m\'as y m\'as parte de la vida diaria, aplic\'andose en situaciones directamente  relacionadas con la vida de los seres humanos. Es un poblema conocido y estudiado que los modelo pueden producir predicciones sesgadas dependiendo a ciertas caracter\'isticas de los datos (\cite{mehrabi2021survey}). Esto es de gran importancia cuando los datos representan seres humanos y la decisi\'on del modelo tiene un impacto directo o indirecto en la vida de un individuo o grupo. No se desea que el algoritmo juzge por el color de piel o g\'enero o regi\'on, pero tampoco que sus resultados dejen de ser relevante. Es necesario entonces que el flujo sea capaz de poder optimizar tanto  relevancia, como  nivel de justeza, un problema multiobjetivo.

Tambi\'en existen casos donde se quiere conocer la \textit{interpretabilidad} de un modelo, que implica entender sin necesidad de ser un experto porque el flujo toma ciertas decisiones; o cuan \textit{robusto} es el sistema, tal que perturbaciones en los datos no cambien radicalmente el resultado de este.

%Existen situaciones donde la \textit{interpretabilidad} que implica saber, sin ser un experto en Aprendizaje de M\'aquina, porque el sistema tom\'o ciertas decisi\'on puede ser importante.
%Tambi\'en existen casos donde es necsario producir sistemas de ML justos, para evitar que el algoritmo de ML tenga alg\'un tipo de sesgo por g\'enero, color de piel u otro. 
%Estos problemas se resolver\'ian muy bien con multiobjetivo; en caso de no optimizar la relevancia y optimizar \'unicamente para justeza o interpretabilidad respectivamente producir\'ia un sistema in\'util debido a sus malas predicciones.
La adici\'on de optimizaci\'on multibojetivo en AutoML abr\'e las puertas tambi\'en a m\'etricas que no tienen sentido usarlas individualemente como \textit{precisi\'on} y \textit{recobrado}.
\textit{Precisi\'on} mide la proproci\'on entre los valores relevantes identificados y todos los valores identificados
mientras que \textit{recobrado} mide la proporci\'on entre los valores relevantes identificados y todos los valores relevantes.
Estas m\'etircas usadas por separadas no son represantivas pues basta para tener un recobrado perfecto seleccionar todos los elementos del conjunto de datos, y una precisi\'on casi perfecta escogiendo un pequeño conjunto de elementos.
Para lograr un balance entre estas se utiliza \textit{f-score}, que relaciona ambas medidas y se puede parametrizar con vector de peso para darle m\'as importancia a una o a otra seg\'un el investigador determine, no obstante, con un sitema AutoML que pueda optimizar m\'ultiples criterios se pueden obtener flujos que optimicen para ambas. Esto se puede extender a cualquier par de m\'etricas, ya no es necesario expresar varias m\'etricas como una s\'ola.

% El problema de distillitaion!:
% Es entrenar una red neuronal mas peque\~na con una red neuronal mas grande ya entrenada. La red mas peque\~na aprede a replicar los resultados de la red m\'as grande en cada nivel.

% Una manera de enfrentarse a eso es AI Fairness 360 implementado por IBM que provee maneras de mitigar los sesgos in el pre-procesamiento, procesamiento y post-procesamiento. El algoritmo trabaja sobre los datos para identificar y tratar los sesgos. 

\section*{Antecedentes}
% Darle vaselina a esto:
% - Partir de que exsite un marco de trabajo respecto a sistemas AutoML: AutoGOAL 
% - Partir de que hay un estudio sobre sesgos en inteligencia artificial
% - Por ende, se quiere trabajar en una propuesta que permita multiobjetivo
En el entorno del grupo de Inteligencia Artificial de la Universidad de La Habana se desarrolla AutoGOAL una herramienta de AutoML que permite obtener modelos optimizados con respecto a una sola m\'etrica.

\section*{Problem\'atica}
% No darle la vuelta al pollo
% Decir cual es el problema: Poder utilizar optimizacion mulitobjetivo dentro de AutoGOAL
Los sistemas AutoML para generar sus soluciones modelan un espacio de b\'usqueda sobre el que realizan un exploraci\'on buscando una soluci\'on \'optima.
Estos espacios son complejos y muchas veces no diferenciables.
Optimzizar sim\'ultanemeante para muchos objetivos require de un algoritmo que sea lo m\'as agn\'ostico posible a este espacio y a la misma vez sea capaz de ofrecer buenos resultados.
Se intenta la resoluci\'on del problema utilizando algoritmos gen\'eticos que cumplen con estas cualidades.

%Los sistema de Aprendizaje de M\'aquina Automatizado basados en algoritmos evolutivos son los que pueden tener el espacio de decision m\'as amplio.
%Se propone añadir multi objetivo para optimizar respecto a varias m\'etricas de evaluaci\'on simult\'aneamente. Se quiere utilizar una implementaci\'on de algoritmos gen\'etico ya que han demostrado ser los mejores en la literatura.
%Al  funcionar los dos con algoritmos evoluacionarios se busca hacer una integraci\'on entre ambos.

\section*{Hip\'otesis}
% La integraci\'on de sistema AutoML genetico con algoritmo genetico
 Hip\'otesis.

 
\section*{Objetivo}
\subsection*{Objetivo General}
Optimizaci\'on Multiobjetivo para sistemas AutoML
\subsection*{Objetivos Espec\'ificos}
\begin{itemize}
    \item Estudiar el estado del arte relacionado con el problema
    \item Identificar un algoritmo multiobjetivo que aproveche la estructura de AutoGOAL y su implementaci\'on de Probabilistic Grammatic Evolution (PGE)
    \item Estudiar diferencias cuando se resuelve un mismo problema con optimizaci\'on multibojetivo
    \item Analizar la efectividad de la soluci\'on
\end{itemize}

\section*{Estructura de la Tesis}
