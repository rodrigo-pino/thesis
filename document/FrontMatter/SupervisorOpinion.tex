\begin{opinion}
En la actualidad el Aprendizaje Automático ha llegado a todas las ramas de la industria, ayudando a resolver un gran número de problemas pero creando la necesidad de un enorme número de expertos para poder utilizar las herramientas adecuadas en cada caso.
En este escenario el AutoML propone una solución ayudando con la selección de forma automática de las mejores soluciones con el problema añadido de que incrementa
el costo computacional ya que tiene que evaluar muchas soluciones para resolver cada problema. Realizando esta tarea cada vez. El área de investigación en que incursiona el estudiante propone un enfoque para que los sistemas de AutoML resolver problemas donde las métricas de evaluación estén más cerca de los requerimientos reales.

El estudiante Rodrigo Daniel Pino en esta investigación se adentra en un tema del estado del arte de gran actualidad y para eso tuvo que utilizar conocimientos de varias asignaturas de la carrera y otros que no son parte del curriculum estandar. Su propuesta implicó la definición de problemas de aprendizaje de máquina en el contexto multiobjetivo y una experimentación con diferentes problemas donde se utilizan métricas multiobjetivo. Además implicó conocer una herramienta de AutoML nueva e incorporar su estrategia para evaluar y comparar sus resultados en la práctica.

Sus resultados resultan muy prometedores, permitiendo abrir la puerta a la optimización multiobjetivo manteniendo las ventajas del AutoML heterogéneo.
Esta mejora es considerable para una herramienta del estado del arte, que ya lograba resultados comparables a las mejores herramientas de AutoML existentes.
Más aún, las estrategias desarrolladas en esta investigación han sido aplicadas solamente a un grupo de problemas y métricas, pero pueden ser extendidos fácilmente a muchos otros.

Para poder afrontar el trabajo, el estudiante tuvo que revisar literatura científica relacionada con la temática así como soluciones existentes y bibliotecas de software que pueden ser apropiadas para su utilización. Todo ello con sentido crítico, determinando las mejores aproximaciones y también las dificultades que presentan.
Todo el trabajo fue realizado por el estudiante con una elevada constancia, capacidad de trabajo y habilidades, tanto de gestión, como de desarrollo y de investigación.
Por estas razones pedimos que le sea otorgada al estudiante  Rodrigo Daniel Pino Trueba la máxima calificación y, de esta manera, pueda obtener el título de Licenciado en Ciencia de la Computación.


\begingroup
  \centering
  \wildcard{Dra. Suilan Estévez Velarde}
  \hspace{1cm}
  \wildcard{Lic. Daniel Vald\'es P\'erez}
  \par
\endgroup
\end{opinion}
