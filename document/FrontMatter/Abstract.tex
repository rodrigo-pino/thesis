\begin{resumen}
	El problema multiobjetivo aplicado a sistemas de Aprendizaje de M\'aquina Automatizado (AutoML) es un \'area poco explorada. Este trabajo define el problema de optimizaci\'on multiobjetivo aplicado a sistemas de AutoML. Propone una soluci\'on generalizada para cualquier tipo de sistema aprovechando que sus componentes trabajan en espacios distintos: el \'espacio de b\'usqueda donde habitan los flujos de Aprendizaje Autom\'atico encontrados y el espacio objetivo donde reside la evalucia\'on de cada uno de estos respecto a diversas m\'etricas.
    El sistema AutoML genera soluciones sobre el espacio de decisi\'on y el algoritmo multiobjetivo las ordena en el espacio objetivo.
    La propuesta se inspira en NSGA-II como algoritmo multiobjetivo. 
    Se realiza una implementaci\'on en AutoGOAL, un sistema de AutoML orientado a resolver el problema de AutoML Heter\'ogeno. Se realizan pruebas midiendo \textit{f-score} contra tiempo de entrenamiento y precisi\'on contra recobrando obteniendo resultados satisfactorios que prueban la adici\'on de optimizaci\'on multiobjetivo a sistemas AutoML un beneficio inmediato.
\end{resumen}

\begin{abstract}
	Resumen en inglés
\end{abstract}
