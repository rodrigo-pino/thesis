\begin{resumen}
	El problema multiobjetivo aplicado a sistemas de Aprendizaje de M\'aquina Automatizado (AutoML) es un \'area poco explorada. Este trabajo define el problema de optimizaci\'on multiobjetivo aplicado a sistemas de AutoML y propone una soluci\'on generalizada para cualquier tipo de sistema.
    Aprovechando los dos espacios que componen a los sistemas AutoML: el espacio de b\'usqueda $\mathcal{X}$ donde se encuentran las soluciones del sistema y el espacio objetivo $\mathcal{Y}$ donde reside la evaluaci\'on de dichas soluciones respecto a ciertas m\'etricas. El sistema AutoML opera sobre $\mathcal{X}$ generando y evaluando las soluciones  y un algoritmo multiobjetivo selecciona las mejores en $\mathcal{Y}$.
    La propuesta se inspira en NSGA-II como algoritmo multiobjetivo para seleccionar las mejores soluciones. 
    Se realiza una implementaci\'on en AutoGOAL, un sistema orientado a resolver el problema de AutoML Heterog\'eneo. Se realizan pruebas midiendo \textit{f-score} contra tiempo de entrenamiento y precisi\'on contra recobrando obteniendo resultados satisfactorios que prueban la adici\'on de optimizaci\'on multiobjetivo a sistemas AutoML un beneficio inmediato.
\end{resumen}

\begin{abstract}
	Resumen en inglés
\end{abstract}
